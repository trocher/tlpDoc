%!TEX root = main.tex
We will now present and define the family of \textit{ternary labelling problems}, to do so, we will be using edge labelling formalism \cite{1,LB_MM}.
The alphabet of a \textit{ternary labelling problem} is $\Sigma = \{0,1,2\}$.
As explained in the work on \textit{binary labelling problems} \cite{1}:
\begin{itemize}
    \item The graphs are bipartite graphs where the nodes are hence partitioned in two colors : black and white
    \item The output of the problem must be a labelling of all the edge with labels from $\Sigma$
    \item The white and black constraints define the problem.
\end{itemize}
\section{General Form}\label{section:general}
A \textit{ternary labelling problem} is a tuple $\Pi = (\wdd,\bdd,W,B)$ where :
\begin{itemize}
    \item $\wdd\in\{2,3,...\}$ is the \textit{white degree}
    \item $\bdd\in\{2,3,...\}$ is the \textit{black degree}
    \item $W$ is the \textit{white constraint}, a set of \textit{white configurations}
    \item $B$ is the \textit{black constraint}, a set of \textit{black configurations}
\end{itemize}
Each \textit{white configuration} is a 3-tuple $(x_0,x_1,x_2)$ where $x_0+x_1+x_2 = \wdd$\\
Each \textit{black configuration} is a 3-tuple $(x_0,x_1,x_2)$ where $x_0+x_1+x_2 = \bdd$\\\\
The problem is solved correctly if :
\begin{itemize}
    \item Each edge is labelled with some label $l\in\Sigma$
    \item For each white, resp. black, node, the tuple $(x_0,x_1,x_2)$, with $x_i$ the number of incident edges labelled with the label $i$, is a configuration contained in W, resp. B
\end{itemize}

\section{Alpha Form}\label{section:alpha}
The general form presented above has been chosen as the internal form of a problem in the classifier, however, we can describe a problem $\Pi$ in a more readable way using regular expressions \cite{1,LB_MM} to represent the sets of configurations (constraints) in the following way : $\Pi = (\wdd,\bdd,W,B)$ where :
\begin{itemize}
    \item $\wdd\in\{2,3,...\}$ is the \textit{white degree}
    \item $\bdd\in\{2,3,...\}$ is the \textit{black degree}
    \item $W$ is the \textit{white constraint}, a set of \textit{white configurations}
    \item $B$ is the \textit{black constraint}, a set of \textit{black configurations}
\end{itemize}
Each \textit{white configuration}  and \textit{black configuration} is a multiset containing exactly $\wdd$ (resp. $\bdd$) labels (not necessarily distinct) from $\Sigma$ also know as regular expressions\\\\
The problem is solved correctly if :
\begin{itemize}
    \item Each edge is labelled with some label $l\in\Sigma$
    \item For each white, resp. black, node, the multiset $\{x_1,x_2, ..., x_k\}$ (with $k=\wdd$, resp. $k=\bdd$) of its incident edges labels is a configuration contained in W, resp. B. (note that just like a set, we do not take care of the order of the elements in a multiset)
\end{itemize}
We also use the following notations for defining configurations :
\begin{itemize}
    \item $[x_1x_2,...x_k] = x1|x_2|...|x_k$
    \item $x^k$ for x repeated k times
\end{itemize}
\begin{exmp}
(Maximal-Matching). The maximal matching problem for $\Delta = \wdd = \bdd = 3$ can be described in the alpha form as :
\begin{itemize}
    \item $W = MO^{2}|P^{3}$
    \item $B = M[OP]^{2}|O^{3}$
\end{itemize}
We can then describe the problem using the general form as
$\Pi = (3,3,W,B)$ with :
\begin{itemize}
    \item $W = \{(0,0,3),(1, 2, 0)\}$
    \item $B = \{(0,3,0),(1,2,0),(1,1,1),(1,0,2)\}$
\end{itemize}
(we did here the following mapping : $M\rightarrow 0, O\rightarrow 1, P\rightarrow 2$)
\end{exmp}
\section{The nodes-as-edges notation}
When $\bdd = 2$, one could find an equivalent problem $\Pi'$ to $\Pi = (\wdd,2,W,B)$ on a bipartite graph $G$ using the following idea \cite{1}:

We define the graph $G'$ as G where all black nodes are edges. This lead $G'$ to be exclusively composed of white nodes where only the ones with a degree $\wdd$ are considered by $\Pi'$.
$\Pi'$ acts as the following on the graph $G'$ instead of $G$:\\
Each of its nodes must labels their ports according to $W$ and, for every two such nodes $u$ and $v$, if there is an edge between them, the 2 labels of the corresponding ports must be in $B$\\

\section{Alphabets}
The alphabet of the white constraint of a given problem $\Pi$ is the subset of the labels used in the white configurations, we denote it $A_{\Pi,w}$. Similarly, the alphabet of the black constraint is denoted $A_{\Pi,b}$\\\\
The effective alphabet of a problem is the union between the alphabet of the white constraint and the one of the black constraint $A_{\Pi} = A_{\Pi,w} \cup A_{\Pi,b} \subseteq \Sigma$. It correspond to the labels used by the problem.
\begin{exmp}
Let $\Pi = (2,3,W,B)$ with :
\begin{itemize}
    \item $W = \{(0,0,2)\}$
    \item $B = \{(0,0,3),(0,2,1)\}$
\end{itemize}
The white alphabet is $\{3\}$, the black alphabet is $\{2,3\}$, the effective alphabet is then $\{2,3\}$
\end{exmp}
\section{Equivalence}
Using the notations introduced in section \ref{section:general} and \ref{section:alpha}, a problem can have multiple representations, depending on the mapping that is done to obtain the general form. Thus, this section states the conditions for two representations of a problem in the general form to be equivalent.\\\\
If X is a set of 3-tuples $(x_0,x_1, x_2)$, we denote $X_{a,b,c}$ the set that contains all the 3-tuples $(x_a,x_b, x_c)$ such that $(x_0,x_1, x_2)\in X$ and $a,b,c$ is a permutation of $(0,1,2) $.\\\\
The following problems have the same complexity up to +/- 1 round : 
\begin{itemize}
    \item $\Pi_0 = (\wdd,\bdd,W,B)$
    \item $\Pi_1 = (\bdd,\wdd,B,W)$
    \item $\Pi_2 = (\wdd,\bdd,W_{a,b,c},B_{a,b,c})$
    \item $\Pi_3 = (\bdd,\wdd,B_{a,b,c},W_{a,b,c})$
\end{itemize}
with $a,b,c \in permutations(0,1,2)$\\
We denote the set of these problems as an \textit{equivalent set}
\begin{exmp}\label{exp_def_1}
The following problem with $\wdd = 2$ and $\bdd = 3$ :
\begin{itemize}
    \item $W = BC|AA$
    \item $B = B(CC|BA|BC)$
\end{itemize}
Can be described as :\\
$\Pi = (2,3,W,B)$ with $(W,B)$ or $(B,W)$ being one of the following tuples:\\\\
$$\begin{array}{cc}
    ( \{(0,1,1),(2, 0, 0)\}, &  \{(0,1,2),(1,2,0),(0,2,1)\} )\\
    ( \{(1,0,1),(0, 2, 0)\}, &  \{(2,0,1),(0,1,2),(1,0,2)\} )\\
    ( \{(1,1,0),(0, 0, 2)\}, &  \{(1,2,0),(2,0,1),(2,1,0)\} )\\
    ( \{(0,1,1),(2, 0, 0)\}, &  \{(0,2,1),(1,0,2),(0,1,2)\} )\\
    ( \{(1,0,1),(0, 2, 0)\}, &  \{(1,0,2),(2,1,0),(2,0,1)\} )\\
    ( \{(1,1,0),(0, 0, 2)\}, &  \{(2,1,0),(0,2,1),(1,2,0)\} )
\end{array}$$
\end{exmp}

\begin{exmp}\label{exp_def_2}
The following problems form a equivalent set $\Pi_k = (2,3,W,B)$ with $(W,B)$ or $(B,W)$ as:\\\\
$$\begin{array}{cc}
    ( \{(0,1,1)\}, &  \{(1,1,1)\} )\\
    ( \{(1,0,1)\}, &  \{(1,1,1)\} )\\
    ( \{(1,1,0)\}, &  \{(1,1,1)\} )
\end{array}$$
\end{exmp}

\section{Representative problem}
The representative problem $\Pi_c$ of a set of equivalent problems will be used to represent them all in order to reduce the database size. $\Pi_c$ is found by sorting each of the white and black lists of configurations of the equivalents problems by alpha-numerical order. Then, the whole equivalent list is sorted and the first problem is taken as representative problem. This ensure that only one problem of each equivalent set is used by the classifier.
\section{Restrictions, Relaxations}
A problem $\Pi_1 = (\wdd,\bdd,W_1,B_1)$ is a \textit{restriction} of a problem $\Pi_2 = (\wdd,\bdd,W_2,B_2)$ (and $\Pi_2$ is a relaxation of $\Pi_1$) if and only if : there is one problem $\Pi'_1$ in the equivalence set of $\Pi_1$ and one problem $\Pi'_2$ in the equivalence set of $\Pi_2$ such that $W_{\Pi'_1}\subseteq W_{\Pi'_2}$ and $B_{\Pi'_1}\subseteq B_{\Pi'_2}$.