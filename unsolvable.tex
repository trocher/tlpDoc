%!TEX root = tlp.tex
\subsection{Unsolvable problems}
The Constraint Reduction algorithms presented previously can show the solvability of a given problem, we will show it here.\\\\
Let's take a problem $\Pi = (\wdd,\bdd,W,B)$. After running the constraint reduction algorithm on $\Pi$, we obtain an equivalent problem $\Pi' = (\wdd,\bdd,W',B')$ we can have two different outputs :
\begin{itemize}
    \item $|W'| = 0$ or $|B'| = 0$ : \\\\
    In this case, it seems pretty obvious that $\Pi' = (\wdd,\bdd,W',B')$ cannot be solved since for either the white nodes, either the black nodes, we cannot assign any configuration, since $\Pi'$ is equivalent to $\Pi$, it appear that $\Pi$ is unsolvable as well.
    \item  $|W'| > 0$ and $|B'| > 0$: \\\\
    In this case, we are going to prove that $\Pi$ must be solvable by showing an algorithm that solve it:
    \begin{itemize}
        \item We elect a white leader in the graph using $GATHER$
        \item The leader choose any configuration in $W$
        \item Each time a white (resp. black) node $u$ in the graph has a complete labelling of its adjacent edges, it send, to each of its neighbors $v_i$, $i\in\wdd$ (resp. $i\in\bdd$) the label of their common edge $(u,v_i)$. $u$ does not does anything else after.
        \item Each time a white (resp. black) node $v$ receive such a label $l$, it find a configuration $c$ in $W$ (resp. $B$) that contains $l$ and labels all its other incidents edges with the other labels of $c$
    \end{itemize}
    At some point, the graph should be fully labelled even if it took $O(n)$ time to do it. Remember that since the graph is a tree, a given un-visited node can only have 1 neighbor that has labelled their common edge.
\end{itemize}