%!TEX root = main.tex
Now that all unsolvable problems have been classified, we know that every problem left must have a $\mathcal{O}(n)$ complexity. In this section, we are going to try to show that  $\Omega(n)$ hold on the complexity of some problems (which would lead the problem to be $\Theta(n)$).
\section[Proving Omega(n)]{Proving $\Omega(n)$}
The binary labelling problem classification described in section \ref{sec:BLP} is implemented and shows that $\Omega(n)$ hold for a lot of problems. Indeed, while some problems $\Pi$ having a global complexity are classified since $|A_\Pi|<3$, some other with $|A_\Pi| = 3$ are classified using the redundancy of their labels. In the case of $\wdd = 3, \bdd = 2$, this leads to the classification of 153 problems having a global complexity.