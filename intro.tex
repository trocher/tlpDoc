%!TEX root = main.tex
\subsection{General Form}
The alphabet of a \textit{ternary labelling problem} is $\Sigma = \{0,1,2\}$\\\\
A \textit{ternary labelling problem} is a tuple $\Pi = (\wdd,\bdd,W,B)$ where :
\begin{itemize}
    \item $\wdd\in\{2,3,...\}$ is the \textit{white degree}
    \item $\bdd\in\{2,3,...\}$ is the \textit{black degree}
    \item $W$ is the \textit{white constraint}, a set of \textit{white configurations}
    \item $B$ is the \textit{black constraint}, a set of \textit{black configurations}
\end{itemize}
Each \textit{white configuration} is a 3-tuple $(x_1,x_2,x_3)$ where $x_1+x_2+x_3 = \wdd$\\
Each \textit{black configuration} is a 3-tuple $(x_1,x_2,x_3)$ where $x_1+x_2+x_3 = \bdd$\\\\
The problem is solved correctly if :
\begin{itemize}
    \item Each edge is labelled with some label $l\in\Sigma$
    \item For each white, resp. black, node, the tuple $(x_1,x_2, x_3)$, with $x_i$ the number of incident edges labelled with the label $i$, is a configuration contained in W, resp. B
\end{itemize}
\begin{exmp}
(maximal-Matching). The maximal matching problem for $\Delta = \wdd = \bdd = 3$ can be described with regular expression as :
\begin{itemize}
    \item $W = MO^{2}|P^{3}$
    \item $B = M[OP]^{2}|O^{3}$
\end{itemize}
We can then describe the problem using the previous notation as
$\Pi = (3,3,W,B)$ with :
\begin{itemize}
    \item $W = \{(0,0,3),(1, 2, 0)\}$
    \item $B = \{(0,3,0),(1,2,0),(1,1,1),(1,0,2)\}$
\end{itemize}
(we did here the following "mapping" : $M\rightarrow 0, O\rightarrow 1, P\rightarrow 2$)
\end{exmp}
\subsection{Nodes-as-edges form}
When $\bdd = 2$, One could see the problem as following :
The black nodes are considered as edges. This lead the tree to be only composed of white nodes where only the one with a degree $\wdd$ are considered.\\
Each of theses nodes must labels their ports according to $W$ and for every two such nodes $u$ and $v$, if there is an edge between them, the 2 labels of the corresponding ports must be in $B$\\

\subsection{Alphabets}
The alphabet of the white constraint of a given problem $\Pi$ is the subset of the labels used in all the white configurations, we denote it $A_{\Pi,w}$. Similarly, the alphabet of the black constraint is denoted $A_{\Pi,b}$\\\\
The effective alphabet of the problem is the union between the alphabet of the white constraint and the black constraint $A_{\Pi} = A_{\Pi,w} \cup A_{\Pi,b} \subseteq \Sigma$. It correspond to the labels used by the problem.
\begin{exmp}
Let $\Pi = (2,3,W,B)$ with :
\begin{itemize}
    \item $W = \{(0,0,2)\}$
    \item $B = \{(0,0,3),(0,2,1)\}$
\end{itemize}
The white alphabet is $\{3\}$, the black alphabet is $\{2,3\}$, the effective alphabet is then $\{2,3\}$
\end{exmp}
\subsection{Equivalence}
As we can see in the example 1.1, the introduced notation can represent a problem, however, it is not the only way to represent it since another "mapping" could have been done leading to another representation of the problem. This subsection does then state the conditions for two representations of the problem to be equivalent.\\\\
If X is a set of 3-tuple $(x_1,x_2, x_3)$, we denote $X_{a,b,c}$ the set that contains all the 3-tuple $(x_a,x_b, x_c)$ such that $(x_1,x_2, x_3)\in X$  $a,b,c$ being a permutation of $(1,2,3) $\\\\
The followings problems have the same complexity up to +/- 1 round : 
\begin{itemize}
    \item $\Pi_0 = (\wdd,\bdd,W,B)$
    \item $\Pi_1 = (\bdd,\wdd,B,W)$
    \item $\Pi_2 = (\wdd,\bdd,W_{a,b,c},B_{a,b,c})$
    \item $\Pi_3 = (\bdd,\wdd,B_{a,b,c},W_{a,b,c})$
\end{itemize}
with $a,b,c \in permutations(1,2,3)$
\begin{exmp}
The following problem with $\wdd = 2$ and $\bdd = 3$ :
\begin{itemize}
    \item $W = BC|AA$
    \item $B = B(CC|BA|BC)$
\end{itemize}
Can be described as :
$\Pi = (2,3,W,B)$ with $(W,B)$ or $(B,W)$ being one of the following tuple:\\\\
\begin{array}{cc}
    $( \{(0,1,1),(2, 0, 0)\}$ &  $\{(0,1,2),(1,2,0),(0,2,1)\} )$\\
    $( \{(1,0,1),(0, 2, 0)\}$ &  $\{(2,0,1),(0,1,2),(1,0,2)\} )$\\
    $( \{(1,1,0),(0, 0, 2)\}$ &  $\{(1,2,0),(2,0,1),(2,1,0)\} )$\\
    $( \{(0,1,1),(2, 0, 0)\}$ &  $\{(0,2,1),(1,0,2),(0,1,2)\} )$\\
    $( \{(1,0,1),(0, 2, 0)\}$ &  $\{(1,0,2),(2,1,0),(2,0,1)\} )$\\
    $( \{(1,1,0),(0, 0, 2)\}$ &  $\{(2,1,0),(0,2,1),(1,2,0)\} )$
\end{array}
\end{exmp}
\subsection{Representative problem}
The representative problem $\Pi_c$ of set of equivalents problems will be used to represent them all in order to reduce the database size. It is found by sorting each of the white and black list of configurations of the problems in the equivalent list by alpha-numerical order and then sorting the list of problems to take the first problem in it.
\begin{exmp}
Considering the previous example,
\end{exmp}
\subsection{Restrictions, Relaxations}
A problem $\Pi_1 = (\wdd,\bdd,W_1,B_1)$ is a \textit{restriction} of a problem $\Pi_2 = (\wdd,\bdd,W_2,B_2)$ (and $\Pi_2$ is a relaxation of $\Pi_1$) if and only if $W_1\subseteq W_2$ and $B_1\subseteq B_2$