%!TEX root = main.tex

\section{Known logarithmic problems}
\subsection{The 3-vertex coloring}
It is known that d-vertex coloring has a complexity $\Theta(log_dn)$ on d-regular trees \cite{DBLP:journals/corr/ChangKP16}, therefore we can classify the 3-vertex coloring problem on 3-regular trees: $$W = \{AB, AC, BC\}, B =\{AAA,BBB,CCC\}$$


\subsection{The 3-edge coloring}
First, we know that d-edge coloring has a complexity $\mathcal{O}(log_dn)$  on d-regular trees with $d\geq 3$ \cite{DBLP:journals/corr/abs-1708-04290}.
We also know that 3-edge coloring has a complexity $\Omega(log_n)$ on 3-regular trees \cite{balliu2019locality}, we can then classify the 3-edge coloring problem on 3-regular trees: $$W = \{AA, BB, CC\}, B=\{ABC\}$$


\section{Logarithmic upper bound using sinkless and sourceless orientation}
From \cite{1} we know that creating a sinkless and sourceless orientation (SSO) in a 3 regular has a $\Theta(log(n))$ complexity \cite{1}. Let use this result to show that some 3 labelling problems have a $\mathcal{O}(log(n))$ complexity. To do that, for each problem, we will show that given any graph with a SSO, there exist a constant algorithm that solve it.\\
The SSO ensure that, on any 3 regular graph, a given node $u$ with a degree 3 has either:
\begin{itemize}
    \item 2 outgoing edges and 1 incoming edge
    \item 1 outgoing edge and 2 incoming edges
\end{itemize}
The former will be denoted $X$ when the latter $Y$
\subsection[(W = (ABC, BCC), B = (AC,BC)]{$W = \{ABC, BCC\}$, $B = \{AC, BC\}$}
\begin{itemize}
    \item We label the nodes of type $X$ with $BCC$ such that the port corresponding to the incoming edge is labelled with $B$
    \item We label the nodes of type $Y$ with $ABC$ such that the port corresponding to the outgoing edge is labelled with $C$
\end{itemize}
The white constraint is respected since there are no other configuration used for any node with a degree 3.
We can see that any oriented edge $(u,v)$ start from a port of $u$ labelled with a $C$ and can either end on a port of $v$ labelled with $A$ or $B$, since the black constraint contains both $AC$ and $BC$ in both of theses cases the configuration is valid.
\subsection[(W = (AAB, BBC), B = (AB,BC)]{$W = \{AAB, BBC\}$, $B = \{AB, BC\}$}


\begin{itemize}
    \item We label the nodes of type $X$ with $BBC$ such that the port corresponding to the incoming edge is labelled with $C$
    \item We label the nodes of type $Y$ with $AAB$ such that the port corresponding to the outgoing edge is labelled with $B$
\end{itemize}
The white constraint is respected since there are no other configuration used for any node with a degree 3.
We can see that any oriented edge $(u,v)$ start from a port of $u$ labelled with a $B$ and can either end on a port of $v$ labelled with $A$ or $C$, since the black constraint contains both $BA$ and $BC$ in both of theses cases the configuration is valid.



\section{Logarithmic upper bound using even orientation}
Again, from \cite{1} we know that creating an even orientation (EO) in a 3 regular has a $\Theta(log(n))$ complexity \cite{1}. Let use this result to show that some 3 labelling problems have a $\mathcal{O}(log(n))$ complexity. To do that, for each problem, we will show that given any graph with a EO, there exist a constant algorithm that solve it.\\
The EO ensure that, on any 3 regular graph, a given node $u$ with a degree 3 has either:
\begin{itemize}
    \item 2 outgoing edges and 1 incoming edge
    \item 0 outgoing edge and 3 incoming edges
\end{itemize}
The former will be denoted $X$ when the latter $Y$
\subsection[(W = (ABC, CCC), B = (AC,BC)]{$W = \{ABC, CCC\}$, $B = \{AC, BC\}$}
\begin{itemize}
    \item We label the nodes of type $X$ with $ABC$ such that the port corresponding to the incoming edge is labelled with $C$
    \item We label the nodes of type $Y$ with $CCC$
\end{itemize}
The white constraint is respected since there are no other configuration used for any node with a degree 3.
We can see that any oriented edge $(u,v)$ start from a port of $u$ labelled with either $A$ or $B$ and end on a port of $v$ labelled with $C$, since the black constraint contains both $AC$ and $BC$ in both of theses cases the configuration is valid.

\subsection[(W = (X0X1X2, X3CC), B = (AC,BC)]{All the problems where : $W = \{X_0X_1X_2, X_3CC\}$, $B = \{AC, BC\}$ with $X_i \in \{A,B\}$ for $i=0,1,2,3$}
\begin{itemize}
    \item We label the nodes of type $X$ with $X_3CC$ such that the port corresponding to the incoming edge is labelled with $X_3$
    \item We label the nodes of type $Y$ with $X_0X_1X_2$
\end{itemize}
The white constraint is respected since there are no other configuration used for any node with a degree 3.
We can see that any oriented edge $(u,v)$ start from a port of $u$ labelled with $C$ and end on a port of $v$ labelled with either $A$ or $B$, since the black constraint contains both $CA$ and $CB$ in both of theses cases the configuration is valid.


\section{Logarithmic upper bound using even orientation and a 3-edge coloring}
Again, from \cite{1} we know that creating an even orientation (EO) in a 3 regular has a $\Theta(log(n))$ complexity \cite{1}, the sames apply to creating a 3-edge coloring. Let use theses result to show that some 3 labelling problems have a $\mathcal{O}(log(n))$ complexity. To do that, for each problem, we will show that given any graph with a EO, there exist a constant algorithm that solve it.\\
The EO and the 3-edge coloring $(R,G,B)$ ensure that, on any 3 regular graph, a given node $u$ with a degree 3 has either:
\begin{itemize}
    \item 2 outgoing edges and 1 incoming edge.
    The coloring of the edges of the node can lead to 3 possibilities:
    \begin{itemize}
        \item $X_1$ : The incoming edge has color R and the 2 outgoing edges has color G and B
        \item $X_2$ : The incoming edge has color G and the 2 outgoing edges has color R and B
        \item $X_3$ : The incoming edge has color B and the 2 outgoing edges has color R and G
    \end{itemize}
    \item 0 outgoing edge and 3 incoming edges, one for each color. We denote theses node $Y$
\end{itemize}
\subsection[(W = (ABC), B = (AA,BC)]{$W = \{ABC\}$, $B = \{AA, BC\}$}
\begin{itemize}
    \item We label the ports of the nodes of type $X_1$ depending on the color of the corresponding edge in the following way : $R \rightarrow C$, $B \rightarrow A$, $G \rightarrow B$, this respect the white constraint since $ABC\in W$
    \item We label the ports of the nodes of type $X_2$ depending on the color of the corresponding edge in the following way : $R \rightarrow C$, $B \rightarrow A$, $G \rightarrow B$, this respect the white constraint since $ABC\in W$
    \item We label the ports of the nodes of type $X_3$ depending on the color of the corresponding edge in the following way : $R \rightarrow C$, $B \rightarrow A$, $G \rightarrow B$, this respect the white constraint since $ABC\in W$
    \item We label the nodes of type $Y$ with $CCC$
\end{itemize}
The white constraint is respected since there are no other configuration used for any node with a degree 3.
We can see that any oriented edge $(u,v)$ start from a port of $u$ labelled with either $A$ or $B$ and end on a port of $v$ labelled with $C$, since the black constraint contains both $AC$ and $BC$ in both of theses cases the configuration is valid.


\section{Logarithmic upper bound using 3-vertex coloring and 3-edge coloring}
Since both 3-vertex coloring an 3-edge coloring have a complexity $\Theta(log_n)$ on 3-regular trees, it would be enough to have an algorithm that run in constant time in 3-regular given a 3-vertex coloring $(X,Y,Z)$ and a 3-edge coloring to prove that a given problem have a $\mathcal{O}(log(n))$ complexity.

We make the coloring greedy in the way that a node is labelled with Y if it has a neighbor X and a node is labelled with Z if it has both a neighbor X and a neighbor Y


\section{Logarithmic upper Bound using RCP(x)}
In the following we are going to show algorithms of complexity $\mathcal{O}(\log{}n)$ for specifics problems.\\
\subsection{$W = \{AB,CC\}$, $B = \{ABC\}$}

\section{Logarithmic upper Bound using RCP(x)}
In the following we are going to show algorithms of complexity $\mathcal{O}(\log{}n)$ for specifics problems.\\
\subsection{$W = \{AA,BC\}$, $B = \{ABC,AAB,BBC\}$}

Since $\wdd = 2$, the white nodes will act here as 'passive' and be considered as edges. Hence, for such an edge between $u$ and $v$, two white nodes, it as to be labelled with 2 labels respectively from $u$ and $v$. The created tuple has to be in the white constraint set to be valid. We will hence talk about port labelling instead of edge labeling\\

We apply the procedure $RCP(3)$ to the input tree G. Denote by $V_1, . . . , V_L$ the resulting decomposition. We process the layers one by one, from layer L to 1.\\
For a node $u$ on the layer i, assuming that all nodes of layers higher than i have labeled the port of their incident edges in a valid manner. We show that nodes of layer i can label the port of their incident edges as well. Recall that only nodes of degree 3 are relevant and all other nodes are unconstrained.\\

By construction, each node on the layer i must have at most 1 neighbor on layers $i+1,i+2,...,L$  and at most 2 neighbors on layers $i,i+1,i+2,...,L$.\\
Hence a node can have either one edge already partially labeled by a node from an higher layer either no edge at all labeled.\\\\
We start by describing what a white node that have no neighbors on an higher level should do :\\
For every neighbors on its level, label the corresponding port with an A, then, for every neighbors on an lower level, label the corresponding port with either a B either a C depending on what is still available to achieve a valid configuration.
\begin{itemize}
    \item \textcolor{red}{0 neighbors on layer i} and \textcolor{blue}{3 neighbor on layer $k<i$}: \textcolor{blue}{BBC}
    \item \textcolor{red}{1 neighbor on layer i} and \textcolor{blue}{2 neighbors on layer $k<i$} : \textcolor{red}{A}\textcolor{blue}{BC}
    \item \textcolor{red}{2 neighbors on layer i} and \textcolor{blue}{1 neighbor on layer $k<i$}: \textcolor{red}{AA}\textcolor{blue}{B}
\end{itemize}
We now describe what a white node that has 1 neighbors on an higher level should do :\\
Label the port of the edge that incident to an higher level node $v$ with $B$ if $v$ has label its port with a $C$, $C$ if $v$ has label it with a $B$.\\
If the node has a neighbors on its level, label the port of the incident edge with A.\\ Then, for every neighbors on an lower level, label the corresponding port with either a B either a C depending on what is still available to achieve a valid configuration.
\begin{itemize}
    \item If the neighbor higher level node labelled its port with $C$:
    \begin{itemize}
        \item \textcolor{green}{1 neighbor on layer $l>i$}, \textcolor{red}{0 neighbors on layer i} and \textcolor{blue}{2 neighbor on layer $k<i$}: \textcolor{green}{B}\textcolor{blue}{BC}
        \item \textcolor{green}{1 neighbor on layer $l>i$}, \textcolor{red}{1 neighbor on layer i} and \textcolor{blue}{1 neighbors on layer $k<i$} : \textcolor{red}{A}\textcolor{green}{B}\textcolor{blue}{C}
    \end{itemize}
    
    \item If the neighbor higher level node labelled its port with $B$:
    \begin{itemize}
        \item \textcolor{green}{1 neighbor on layer $l>i$}, \textcolor{red}{0 neighbors on layer i} and \textcolor{blue}{2 neighbor on layer $k<i$}: \textcolor{blue}{BB}\textcolor{green}{C}
        \item \textcolor{green}{1 neighbor on layer $l>i$}, \textcolor{red}{1 neighbor on layer i} and \textcolor{blue}{1 neighbors on layer $k<i$} : \textcolor{red}{A}\textcolor{blue}{B}\textcolor{green}{C}
    \end{itemize}
\end{itemize}
Note that the choices of the nodes on the same level do not conflict with each other, as all edges between nodes of level i are labeled with $AA$. Thus we can safely label each layer in O(1) rounds.