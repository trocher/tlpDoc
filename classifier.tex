%!TEX root = main.tex
\section{Generation of the problems}
\subsection{Non-reduced problem set}
We first generate all the possible white (resp. black) configurations by computing all the 3-tuple $(x_1,x_2, x_3)$ such that the 3 integer sum to $\wdd$ (resp. $\bdd$).
The powerset of this set will then be the set of all the possible \textit{white (resp. black) constraints}.
The cartesian product between the set of white constraints and the set of black constraints represent all possible problems. The following table represent the number of problems depending on $\wdd$ and $\bdd$
\begin{center}
\begin{tabular}{ | c | c | c | c |}
 \hline
 \diagbox{$\wdd$}{$\bdd$} & 2 & 3 & 4 \\ 
 \hline
 2 & 4096 & 65536 & 2097152\\
 \hline
 3 &  & 1048576 & 33554432\\
 \hline
 4 &  &  &  1073741824\\
\hline
\end{tabular}
\end{center}
\subsection{Representative problem set}
In order to reduce the data set and avoid to classify problems that are equivalents, for each equivalence set of problems, we can only keep the representative problems in the problem set. This lead to the following number of problems presented in the following table, still depending on $\wdd$ and $\bdd$
\begin{center}
\begin{tabular}{ | c | c | c | c |}
 \hline
 \diagbox{$\wdd$}{$\bdd$} & 2 & 3 & 4 \\ 
 \hline
 2 & 430 & 11456 & 353664\\
 \hline
 3 &  & 88792 & ?\\
 \hline
 4 &  &  &  ?\\
\hline
\end{tabular}
\end{center}

We can observe a set of problems 6 times smaller when $\wdd \neq \bdd$ (due to removing 5 of the 6 different permutations of the configurations tuples to only keep the characteristic problem) and 12 times smaller when $\wdd=\bdd$ due to the additional white/black symmetry

\subsection{The constraint reduction algorithm}\label{sec:CR}

\subsubsection{The algorithm}
We are going to introduce an algorithms that, when applied to each of the problems of the data-set, will reduce its size again considering that a problem with a "useless" configurations has an equivalent complexity of the same problem reduced to only its "useful" configurations.\\

The algorithm consist in removing all the white and black configurations of a problem that cannot be used by any node by the problem. It appear that it can be used to show the solvability of a problem as it will be showed later, however it is also useful to have a better understanding of a problem still unclassified.
\begin{itemize}
    \item The algorithm take as input a problem $\Pi = (\wdd,\bdd,W,B)$
    \item We start by looking at the set $S$ of labels that are in $A_{\Pi}$ but not in  $A_{\Pi,w} \cap A_{\Pi,b}$.
    \item For each labels $l$ present in $S$ we know that no node can use a configuration containing $l$ since one of its neighbors would have to have a corresponding configurations also containing $l$ which is impossible by construction of $S$.
    \item We hence know that all the white or black configurations containing any labels of $S$ will never be used, we can then safely remove them from the white and black constraint sets without modifying the problem.
    \item It is now possible to repeat this procedure on the new white and black constraints sets obtained until either one of them is empty or $S$ is empty
\end{itemize}
\begin{exmp} Constraint reduction algorithm on $\Pi = (3, 2, \{AAA, ABA\}, \{AA, CB, CC\})$

In the first iteration, we have $S = \{C\}$ since there is no configuration with $C$ in $W$ while there is one in $B$.
This lead us to remove both $CC$ and $CB$ from $B$ since no white node could properly label if one of its black neighbor would label their common edge with a $C$.
The new constraints set are :
$$W' = \{AAA, ABA\}, B' = \{AA\}$$
By running another iteration of the algorithm we get $S = \{B\}$ since $CB$ has been removed from $B$ in the previous iteration.
This hence lead to remove $ABA$ from $W$ since this time a black node could not properly label its edge if it is adjacent to a white node that label their common edge with a $B$.
The new constraints set are :
$$W' = \{AAA\}, B' = \{AA\}$$
This time, the algorithm return $S = \{\}$, it hence finish and return the following problem $\Pi' = (3, 2, \{AAA\}, \{AA\})$, which is equivalent to $\Pi$.
\end{exmp}

\begin{exmp} Constraint reduction algorithm on $\Pi = (3, 2, \{ABB, ABA, BBB\}, \{AA, CB, CC\})$

In the first iteration, we have $S = \{C\}$ since there is no configuration with $C$ in $W$ while there is one in $B$.
This lead us to remove both $CC$ and $CB$ from $B$ since no white node could properly label if one of its black neighbor would label their common edge with a $C$.
The new constraints set are :
$$W' = \{ABB, ABA, BBB\}, B' = \{AA\}$$
By running another iteration of the algorithm we get $S = \{B\}$ since $CB$ has been removed from $B$ in the previous iteration.
This hence lead to remove $ABB$, $ABA$ and $BBB$ from $W$ since this time a black node could not properly label its edge if it is adjacent to a white node that label their common edge with a $B$.
The new constraints set are :
$$W' = \{\}, B' = \{AA\}$$
$W'$ is empty, the algorithm hence finish and return the following problem $\Pi' = (3, 2,\{\}, \{AA\})$, equivalent to $\Pi$, this problem is obviously not solvable, the utility of the constraint reduction in proving the solvability of a problem will be showed later when classifing the unsolvable problems.
\end{exmp}
\subsubsection{Improvement on the size of the problem set}

Using the algorithm reduce the number of problems to classify since a lot of them are reduced to some same problems, this lead to the following number of problems presented in the table, again depending on $\wdd$ and $\bdd$:
\begin{center}
\begin{tabular}{ | c | c | c | c |}
 \hline
 \diagbox{$\wdd$}{$\bdd$} & 2 & 3 & 4 \\ 
 \hline
 2 & 248 & 7962 & \\
 \hline
 3 &  & 81694 & ?\\
 \hline
 4 &  &  &  ?\\
\hline
\end{tabular}
\end{center}
\section{Classification of a problem}
We are going to classify the problems into the 5 following categories : \textit{constant, iterated-logarithmic, logarithmic, global, unsolvable}. To classify a problem $\Pi$ in a category $c$, one should show that $\Pi$  have $c$ as lower bound on its complexity (in other world, it is not possible to solve $\Pi$ in $o(c)$ and that $\Pi$ has $c$ as upper bound on its complexity ($\Pi$ is solvable in $O(c)$). Since a lot of problems are relaxations or restriction of other problems, we are going to use this fact to "propagate" lower and upper bound in the following way.
\subsection{Giving a lower bound to a problem}
Suppose we have some lower-bound of the complexity $l$ of a problem $\Pi$. We should follow the following steps:
\begin{itemize}
    \item If the problem had a previously attributed lower-bound on the problem that was smaller than the new one, replace it by $l$ if it had a previously attributed upper-bound that is smaller than the new lower-bound, there must be an error.
    \item For all the problems that are a restriction of any problems of the set of equivalent problems of $\pi$, set its lower-bound to $l$
\end{itemize}
\subsection{Giving a upper bound to a problem}
Suppose we have some upper-bound of the complexity $u$ of a problem $\Pi$. We should follow the following steps:
\begin{itemize}
    \item If the problem had a previously attributed upper-bound on the problem that was bigger than the new one, replace it by $u$
    \item For all the problems that are a relaxation of any problems of the set of equivalent problems of $\pi$, set its upper-bound to $u$
\end{itemize}
Setting a complexity $c$ to a problem is then similar to applying both of theses process to the problem.